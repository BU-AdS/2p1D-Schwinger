% ------------------------------------------------------------------
\documentclass[12 pt]{article}
\newcommand\ignore[1]{}
\usepackage[left]{lineno}
\usepackage{amsmath}
\usepackage{amssymb}
\usepackage{cancel}
 \usepackage{graphicx}
\usepackage{braket}
\usepackage{authblk}
\usepackage{caption,subcaption}
\usepackage{comment}
\usepackage{enumitem}
\pagestyle{plain}
\pagenumbering{arabic}
\usepackage{color}
\newcommand{\rcb}[1]{\textcolor{blue}{  #1 }}
\newcommand{\rcbfoot}[1]{\textcolor{red}{**\footnote{\textcolor{blue}{ \sc COMMENT REMOVE LATER *** #1 ***}}}}
\pdfpagewidth 8.5 in
\pdfpageheight 11 in
\setlength{\parindent}{0 mm}
\setlength{\parskip}{10pt}
\setlength{\abovedisplayskip}{0 pt}
\setlength{\belowdisplayskip}{0 pt}

\usepackage{amsmath}
\usepackage{amssymb}
\usepackage{graphicx}
\usepackage[margin = .75 in]{geometry}
\usepackage[pdftex, pdfstartview={FitH}, pdfnewwindow=true, colorlinks=false, pdfpagemode=UseNone]{hyperref}

% Laziness shortcuts

\newcommand\dd{\partial}
\newcommand{\nn}{\nonumber \\}
\newcommand\be{\begin{equation}}
\newcommand\ee{\end{equation}}
\newcommand\bea{\begin{eqnarray}}
\newcommand\eea{\end{eqnarray}}
\newcommand{\<}{\langle}
\renewcommand{\>}{\rangle}
\newcommand\half{ \textstyle {\frac{1}{2}}}
% ------------------------------------------------------------------
\bibliographystyle{unsrt}

% ------------------------------------------------------------------
\begin{document}

\begin{center}
 \Large \bf 2D U(1) Gauge Slab Action
\end{center}

\section{Massless $\phi^4$ Theory}
Consider the 1D central line of a 1D $\phi^4$ theory. One may ask the question "what is the effective action on the central line if we integrate out the action from the non-central lines?"

We shall refer to coordinates on the $x$- plane simply as $x$ and coordinates in the extra dimension as $s$. Hence, a lattice point will in general be given by $(x,s)$. The full action given in terms of the discrete Laplace operator is
\be
S =  \frac{1}{2} \sum_{x,s} (\phi(x+a,s) - \phi(x,s))^2 + \frac{1}{2}\sum_{x}\sum_{s=-L_{s}/2}^{L_s/2-1}(\phi(x,s+a) - \phi(x,s))^2
\ee
so we are asking the question 



\section{Non-compact Gaussian Action}

Here is the 2d $L^2$ action  with $L_s + 1$  slices: $s = 0, \pm 1, \cdots L_s/2$.
%
\be
S =  \frac{1}{4} \sum_{x,s}   \sum_{\mu,\nu} F_{\mu \nu}(x,s) F_{\mu
  \nu}(x,s)  + \frac{1}{2} \sum_{\mu,\nu} E_\mu(x,s) E_\mu(x,s)  
\ee
where 
\be
F_{\mu \nu}(x,s)  = \Delta_\mu \theta_\nu(x,s) - \Delta_\nu \theta_\mu(x,s) \nn
 = ( \theta_\nu(x +\mu,s) - \theta_\nu(x,s) ) - ( \theta_\mu(x +\nu,s) - \theta_\mu(x,s) )
\ee
and 
\be
E_\mu(x,s) = \Delta_s \theta_\mu(x,s) = \theta_\mu(x,s+1)  -\theta_\mu(x,s) 
\ee
Therefore,
\be
S =  \frac{1}{2} \sum_{x,s}  \sum_{\mu <\nu} ( ( \theta_\nu(x +\mu,s)
- \theta_\nu(x,s) ) - ( \theta_\mu(x +\nu,s) - \theta_\mu(x,s) ))^2
+ \frac{1}{2} \sum_{x,s} \sum^{L_s/2 -1}_{s= -L_s/2} (\theta_\mu(x,s+1)  -\theta_\mu(x,s) )^2
\ee
We can go to momentum space by a unitary transformation :
\be
\theta_\mu(x,s) =  \frac{1}{(2 \pi)^2}\int^\pi_{-\pi} d^2k   e^{i x k}\widetilde\theta_\mu(k,s) \quad 
\mbox{and} \quad  \widetilde \theta_\mu(k,s) = \frac{1}{L}\sum_{x \in Z}  e^{-i x k} \theta_\mu(x,s) 
\ee
and 
\be
\Delta_\mu \theta_\nu(x,s)
= \frac{1}{L}\sum_k(e^{ik_\mu} - 1)  e^{i x k} \widetilde
\theta_\nu(k,s) 
\ee

or defining $ (e^{ik_\mu} - 1) =  i \hat k_\mu $ this  gives,
\bea
S &=&  \frac{1}{2} \sum_{k,s}\sum_{\mu <\nu} [ \hat k^*_\mu  \widetilde \theta^*_\nu(k,s) -
\hat k^*_\nu   \widetilde \theta^*_\mu(k,s) ]  [ \hat k_\mu  \widetilde \theta_\nu(k,s) -
\hat k_\nu   \widetilde \theta_\mu(k,s) ]  \nn
&+& \frac{1}{2} \sum^{L_s/2 -1}_{s= -L_s/2}  \sum_{k,\mu} (\widetilde\theta^*_\mu(k,s+1) -\widetilde\theta^*_\mu(k,s))(\widetilde\theta_\mu(k,s+1)  -\widetilde\theta_\mu(k,s) )
\eea
(Note in 2D  $\mu = x$, and $\nu = y$ so there is no sum at all!) 
The quadratic form is 
\be
S = \frac{1}{2} \widetilde \theta^*_\mu(k,s)
M_{\mu\nu}(k)\theta_\nu(k,s)  - \frac{1}{2} [\widetilde
\theta^*_\mu(k,s) \theta_\mu(k,s+1) + \widetilde
\theta^*_\mu(k,s+1) \theta_\mu(k,s)]
\ee
Since it is of course diagonal in k, the sum over $k$ implicit. 

We now integrate all {\bf but} the zero-th central slice. Formally separating
calling the thetas on the midels slide $\widetilde \theta(k,0) \equiv
\widetilde \theta_\mu(0) $ we don the 
 integral for over the others  $\Theta_{\mu s}(k) = \widetilde \theta_\mu(k,s \ne 0)$'s; to get the effective action in
$k-space$:
%
\bea
&& e^{\textstyle - S_{eff}} \nn
 &=& e^{\textstyle -\frac{1}{2} \widetilde \theta^*_\mu(k,0)
M_{\mu\nu}(k) \widetilde \theta_\nu(k,0) }  \nn &\times &\int 
d^2\Theta_{\mu s}(k) 
e^{  \textstyle -\frac{1}{2}  \Theta^\dag_{\mu s}(k) G^{ss'}_{\mu\nu} (k) 
  \Theta_{s'\nu}(k)  + \frac{1}{2}[\widetilde \theta^*_\mu(k,0) (
  \Theta_{1,\mu}(k) + \Theta_{-1,\mu}(k))  +(\Theta^\dag_{1,\mu} + \Theta^\dag_{-1,\mu} )\widetilde \theta_\mu(k,0)]}\nn
&=& e^{\textstyle -\frac{1}{2} \widetilde \theta^*_\mu(k,0)
M_{\mu\nu}(k) \widetilde \theta_\nu(k,0)  +  \frac{1}{2} \widetilde
\theta^*_\mu(k,0) ([1/G(k)]^{11}_{\mu\nu} (k)  +[1/G(k)]^{-1-1}_{\mu\nu} (k)]\widetilde \theta_\nu(k,0) } 
\eea
%
\end{document}